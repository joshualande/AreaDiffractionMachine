One of the most common types of analysis of diffraction data
is to perform an intensity integration in $Q$. This will 
create a plot of average intensity as a function of $Q$.
Since powder diffraction procedures cones of light, this means
that the intensity should be uniformly large for some $Q$
values and uniformly low for others, leading to $Q$ values
where the intensity sharply peaks. The $Q$ values that lead 
to these peaks can be used to learn structural information
about the crystals that are being diffracted. So in principle,
using the transformations just described, it should be easy
to convert all of the pixel coordinates $(x_d,y_d)$ into
$Q$ values and then plot average intensity as a function of
$Q$. The only problem we would face is that in order to do
the transformation, we would need to know the values of the
the parameters that characterize an experiment. These are
$x_c$, $y_c$, $d$, $\lambda$, $\alpha$, $\beta$, and 
$R$.\footnote{The pixel scale $ps$ is usually know in advance 
as a uniform property of the detector being used.} Calibration
then is the process used to find what we will now call the
calibration values.

\subsection{The Calibration Algorithm}\index{calibration}

Although in principle all the calibration values could be
experimentally measured, in practice they can not be
directly measured to an acceptable level of precision. 
Instead, a standard calibration procedure is used to 
infer these values from real diffraction data. The 
trick to doing this calibration is to image a standard
while performing the diffraction analysis of an 
unknown sample. Assuming that the diffraction machine
was not changed between the collection of the 
standard crystal and the diffraction of the unknown sample, 
the calibration data corresponding to the two 
images will be the same. So, if we can figure out the
calibration values of the standard crystal, we can
use these values when analyzing the unknown crystal.
This is exactly what is done in practice.

What it means to use a standard crystal is to know the
particular $Q$ values for which the crystal preferentially
scatters light. With this information, and the 
calibration values for some particular experiment,
we could in principle figure out exactly what diffraction
pattern we should find. This do this, we could, for
each $Q$ value, vary $\chi$ and calculate the 
$(x_d,y_d)$ coordinate corresponding to that $(Q,\chi)$
pair. After using enough $\chi$ values, we would be 
able to fill in the rings as they would show up on 
the detector.

In fact, my program can do just this. If you load in 
a set of $Q$ values (more about this in section 
\ref{TheQValues})\index{$Q$ Values}
and then put into the program some calibration values,
and then push the \gui{Draw Q Values?} check box, 
you can then see what the particular diffraction
image would have shown up on the detector. 
(NEED TO SHOW EXAMPLE HERE)

Being able to do this still leaves us with a hard 
problem to solve. For particular calibration values,
we can eaisly calcualte what the defraction pattern 
should look like. But what we really know is what
the calibration values are for the known diffraction
pattern of a standard crystal. In order to perform the
real calibration, then, we can vary the calibration 
values until they make the mattern that can be calculated
to show up to match the pattern that was actually 
captured. The process of image calibration then is a 
procedure to `fit' the calibraiton values to a
diffraciton patter with known $Q$ values.

\subsection{The Fitting} 

In order for the fitting algorith to work, the program must already have an initial guess of the real calibragion parameters. This initial guess does not have to be prfect, but it has to be close enough. The algorith them requires a list of the known $Q$ values and it furthermore requires a range of $Q$ values 






\begin{SCfigure}
\centering
% Generated with LaTeXDraw 1.9.5
% Sat Apr 26 21:57:04 EDT 2008
% \usepackage[usenames,dvipsnames]{pstricks}
% \usepackage{epsfig}
% \usepackage{pst-grad} % For gradients
% \usepackage{pst-plot} % For axes
\scalebox{1} % Change this value to rescale the drawing.
{
\begin{pspicture}(0,-4.41)(10.381406,4.41)
\pscircle[linewidth=0.04,dimen=outer](4.56,0.23){1.3}
\pscircle[linewidth=0.04,dimen=outer](4.451348,0.027117867){3.830909}
\psline[linewidth=0.04cm,arrowsize=0.1529cm 2.0,arrowlength=1.4,arrowinset=0.2]{->}(4.66,0.31)(8.76,2.91)
\usefont{T1}{ptm}{m}{n}
\rput(8.728282,3.72){\psframebox*[framesep=0, boxsep=false,fillcolor=white] {Constant $\chi$ Slice}}
\rput{-10.244087}(0.04480584,0.8143853){\pscircle[linewidth=0.04,linestyle=dashed,dash=0.17638889cm 0.10583334cm,dimen=outer](4.5651674,0.15725891){1.72815}}
\pscircle[linewidth=0.04,linestyle=dashed,dash=0.17638889cm 0.10583334cm,dimen=outer](4.56,0.21){0.8}
\pscircle[linewidth=0.04,linestyle=dashed,dash=0.17638889cm 0.10583334cm,dimen=outer](4.543829,0.0809504){3.2429092}
\pscircle[linewidth=0.04,linestyle=dashed,dash=0.17638889cm 0.10583334cm,dimen=outer](4.41,0.0){4.41}
\usefont{T1}{ptm}{m}{n}
\rput(8.951406,2.12){\psframebox*[framesep=0, boxsep=false,fillcolor=white] {$Q_2+\Delta Q_2$}}
\usefont{T1}{ptm}{m}{n}
\rput(7.911406,1.84){\psframebox*[framesep=0, boxsep=false,fillcolor=white] {$Q_2$}}
\usefont{T1}{ptm}{m}{n}
\rput(7.811406,1.52){\psframebox*[framesep=0, boxsep=false,fillcolor=white] {$Q_2-\Delta Q_2$}}
\usefont{T1}{ptm}{m}{n}
\rput(6.891406,0.84){\psframebox*[framesep=0, boxsep=false,fillcolor=white] {$Q_1+\Delta Q_1$}}
\usefont{T1}{ptm}{m}{n}
\rput(5.8714066,0.48){\psframebox*[framesep=0, boxsep=false,fillcolor=white] {$Q_1$}}
\usefont{T1}{ptm}{m}{n}
\rput(6.011406,0.04){\psframebox*[framesep=0, boxsep=false,fillcolor=white] {$Q_1-\Delta Q_1$}}
\psbezier[linewidth=0.04](6.8,1.67)(7.22,2.05)(6.9,2.53)(7.2,2.71)(7.5,2.89)(8.04,2.37)(8.32,2.63)
\psbezier[linewidth=0.04](4.94,0.49)(5.258421,0.81393445)(5.02,1.15)(5.2842107,1.2965574)(5.548421,1.4431148)(5.88,1.05)(6.1,1.23)
\end{pspicture} 
}


\caption{Here is a schematic diagram of the peak finding 
algorithm. For a particular $\chi$ slice, my code fits 
Gaussian along the line to find the peaks.}
\label{Fitting}
\end{SCfigure}


\subsection{Performing Image Calibration}

\subsection{The $Q$ File Format}\label{TheQValues}
